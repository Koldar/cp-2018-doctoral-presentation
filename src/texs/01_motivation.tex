\section{Introduction}

\subsection{Introduction}
\begin{frame}{Introduction}
	\begin{itemize}
		\item In constraint-based temporal reasoning, most techniques deal with static or incremental consistency, but \textbf{decremental consistency} has not been investigated yet:
		\begin{itemize}
			\item Given an \textit{in}consistent CSP and a sequence of constraint relaxations, it is the problem of determining if the revised CSP becomes consistent after each relaxation
		\end{itemize}
		
		\item \textbf{Goal}: efficient consistency checking algorithms exploiting decremental problem to amortize complexity over constraint relaxations;
		
		\item \textbf{Results}: developed new decremental algorithms for PA and ORD-Horn subalgebra; experimental results show substantial time gains can be achieved over the static algorithms.
	\end{itemize}
\end{frame}

\subsection{Motivation}
\begin{frame}{Motivation}
	\framesubtitle{A simple example}
	\begin{itemize}
		\item Applications merging qualitative temporal constraints (expressing preferences, beliefs or noisy information) may lead to an \textit{in}consistent CSP which needs to be revised.
	\end{itemize}
	
	\begin{columns}
		\begin{column}{0.5\textwidth}
			\centering
			$<$-constraint graph
		\end{column}
		\begin{column}{0.5\textwidth}
			\centering
			A solution after relaxations
		\end{column}
	\end{columns}	
	
	\begin{columns}
		\column{0.5\textwidth}
			\begin{figure}
				\centering
				\includeTikz{touristExample.ia.before.6}[0.5]
			\end{figure}
		\column{0.5\textwidth}
			\begin{figure}
				\centering
				\includeTikz{touristExample.ia.before.6.solution}[0.5]
			\end{figure}
	\end{columns}
	
	\begin{table}
		\footnotesize
		\centering
		\begin{tabular}{l|ccccccc}
		\hline
		Acronym	& Mu		& Tw	& Ch		& Lu	& Te		& Ca		& Ga		\\ \hline
		Name	& Museum	& Tower	& Church	& Lunch	& Temple	& Castle	& Garden	\\ \hline
		\end{tabular}
	\end{table}
	
\end{frame}